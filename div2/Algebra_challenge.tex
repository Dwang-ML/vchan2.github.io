\documentclass{article}
\usepackage{amsmath,amssymb}
\usepackage{fullpage}
\usepackage{enumitem}
\usepackage{hyperref}
\title{Holiday presents}
\author{Vincent Chan}
\date{\today}

\begin{document}

\begin{center}
\textbf{Algebraic ``Facts'' Challenge}
\end{center}

Here are the properties of integers discussed in class:
\begin{enumerate}
	\item Additive associativity: for any elements $a,b,c$,
	\[
		a+(b+c) = (a+b)+c
	\]
	\item Additive commutativity: for any elements $a,b$,
	\[
		a+b = b+a
	\]
	\item Additive identity: there exists an element $z$ such that for any element $a$,
	\[
		a+z = a
	\]
	We denote $z$ as $0$.
	\item Additive inverse: for any element $a$, there exists a element $b$ such that 
	\[
		a+b = 0
	\]
	We denote $b$ as $-a$.	
	\item Multiplicative associativity: for any elements $a,b,c$,
	\[
		a(bc) = (ab)c
	\]
	\item Multiplicative commutativity: for any numbers $a,b$,
	\[
		ab = ba
	\]
	\item Multiplicative identity: there exists an element $u$ such that for any element $a$,
	\[
		au = a
	\]
	We denote $u = 1$.
	\item Distributivity: for any elements $a,b,c$,
	\[
		a(b+c) = ab+ac
	\]
\end{enumerate}

Prove the following ``facts'' using the above properties:
\begin{enumerate}
\item 0 is the unique additive identity.
\item For any elements $a,b,c$: if $a+b = a+c$, then $b=c$.
\item Additive inverses are unique.
\item For any element $a$: $0a = 0$ (So 0 times anything is 0.)
\item For any elements $a,b$: $(-a)b = -(ab)$ (In particular, a negative times a positive is negative.)
\item For any elements $a,b$: $(-a)(-b) = ab$ (In particular, a negative times a negative is positive.)
\end{enumerate}

\end{document}